\vspace*{6cm}
{\setlength{\parskip}{-0.5cm}
\chapter{Conclusiones y recomendaciones}
}
\newpage

En este capítulo se concluyen los resultados obtenidos del trabajo y se dan recomendaciones para futuras investigaciones.

{\setlength{\parskip}{0cm}
\section{Conclusiones}

\begin{itemize}
    \item Se concluyó que las 3R de la ecología son necesarias para la defensa del medio ambiente, y que la disminución de la emisión de gases de efecto invernadero y de residuos sólidos, el ahorro de la energía, la reducción de la deforestación, entre otros, son algunos de los grandes beneficios de estos principios al medio ambiente.
    
    \item Se evidenció que la mayoría de la población no ponía en práctica o desconocía algunas de las estrategias de las 3R de la ecología. Sin embargo, las personas que conformaron la muestra seleccionada estuvieron de acuerdo en que su rutina del manejo de desechos sólidos podía mejorarse para minimizar los daños colaterales por la generación de los mismos. También mostraron tener algunos hábitos con respecto a la reutilización del plástico, vidrio, cartón, papel y desechos orgánicos. 
    
    \item Se logró captar la atención de la muestra a través de la campaña divulgativa. También su interés en contribuir en la iniciativa de aplicar estrategias propuestas acerca de estos principios del desarrollo sustentable. En este sentido, se observó un cambio positivo en sus hábitos de manejo de los desechos sólidos.
    
    \item Se observó, mediante el seguimiento realizado, que algunas de las estrategias propuestas para el empleo de las 3R de la ecología fueron implementadas con mayor facilidad por la muestra, como reutilizar frascos para almacenar objetos y evitar el uso de platos, cucharas y vasos desechables. Por el contrario, hubo otras que fueron poco aplicadas como, usar botellas plásticas como macetas para plantas y evitar comprar artículos empaquetados con exceso de plástico.
\end{itemize}
}

\newpage

{\setlength{\parskip}{0cm}
\section{Recomendaciones}

\begin{itemize}
    \item Extender la información acerca de los beneficios sobre la aplicación de las estrategias planteadas en la campaña divulgativa para el empleo de las 3R de la ecología a todo el conjunto residencial. 
    
    \item Hacer el seguimiento de la implementación de las estrategias propuestas con mayor frecuencia.

    \item Aplicar las estrategias propuestas en todo el conjunto residencial.
    
    \item Formar brigadas ecológicas en el sector a fin de que sean estas las que promuevan la aplicación de las 3R en el área de estudio.
\end{itemize}
}

\newpage