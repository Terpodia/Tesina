{\setlength{\parskip}{-0.5cm}
\chapter*{Introducción}

La elevada generación de residuos sólidos y su manejo inadecuado son una de las grandes problemáticas ambientales y de salud que se han acentuado en los últimos años debido al incremento de la población y a los patrones de producción y consumo. La basura no solo genera una desagradable imagen, sino que contamina el suelo, el agua y el aire, además, para su confinamiento ocupa grandes espacios, por lo que se ha convertido en un problema social y de salud pública.
}

Con  el objetivo de reducir las consecuencias ocasionadas por esta problemática, la organización ecologista Greenpeace, promueve, con el fin de fomentar el consumo responsable y el manejo adecuado de los desechos, las 3R de la ecología, que son una propuesta sobre hábitos de consumo. El concepto hace referencia a la búsqueda de estrategias de gestión de residuos más sostenibles con el medio ambiente, en particular, dando prioridad a la reducción de la cantidad de residuos generados.

En este sentido, el interés de esta investigación es generar un impacto positivo en la comunidad de la Urbanización Guaraguao de Puerto La Cruz a través de la divulgación de información para la aplicación de las 3R de la ecología en el ámbito doméstico.

Este trabajo consta de un primer capítulo en donde se plantea el problema y su justificación, y se enuncian los objetivos de la investigación. En el segundo capítulo se plantean las bases teóricas que la sustentan. En el tercer capítulo se explica cuál fue la metodología aplicada. En el cuarto capítulo se presentan los resultados y se realiza un análisis de los mismos. Y por último, en el quinto capítulo se realizan conclusiones y recomendaciones con respecto al tema tratado durante la investigación.

\newpage