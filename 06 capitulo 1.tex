\vspace*{6cm}
{\setlength{\parskip}{-0.5cm}
\chapter{El Problema}
}
\newpage

En este capítulo se explican de manera detallada el planteamiento del problema, el objetivo general y los específicos, y la justificación.

{\setlength{\parskip}{0cm}

\section{Planteamiento del problema}

Casi todos los países del mundo se han replanteado estilos de desarrollo diferentes con una orientación más cuidadosa con el ambiente. La medida más empleada y conocida para cumplir con esta meta es el reciclaje, el cual consiste en el proceso de recolección y transformación de materiales para convertirlos en nuevos productos, y que de otro modo serían desechados como basura.
}
Sin embargo, el concepto y la actividad del reciclaje va mucho más allá de las cuestiones técnicas y económicas, implica un esfuerzo colectivo de toda la sociedad: desde los hábitos ciudadanos en la separación de residuos para facilitar el ciclo de reciclaje, hasta las administraciones públicas que deben facilitar ese servicio. 

Además, en los países más desarrollados se suele combinar el reciclaje con otras 2 buenas prácticas para mejorar ampliamente su efectividad, formando así las 3R de la ecología: reducir y reutilizar. Reducir contribuye a la minimización de residuos producidos a diario por la sociedad, consumiendo de forma responsable. Por otra parte, reutilizar permite prolongar el tiempo de vida de los materiales, dándoles el máximo provecho posible antes de desecharlos.

En Venezuela, aún cuando se reconocen diversos problemas en materia ambiental, se cataloga a la basura como uno de los principales, debido a su incidencia en la población de cualquier estrato socioeconómico. Actualmente, no se ha inculcado en las personas hábitos de las 3R de la ecología, por lo que las mismas no aplican las normas para la clasificación de los desechos sólidos y su reducción para preservar los recursos naturales y el ambiente.

El estado Anzoátegui no es la excepción ante el desconocimiento de estas prácticas antes mencionadas dado que no se hace énfasis en una cultura de buena disposición y de poca generación de desechos y residuos, sino que por el contrario se genera en promedio por persona cerca de un kilogramo diario de desechos considerados como no reutilizables por el ciudadano común.

Específicamente, en la Urbanización Guaraguao de Puerto La Cruz, los habitantes tienen un escaso conocimiento de los métodos antes mencionados y no llevan a cabo prácticas como una buena clasificación de desperdicios para que luego puedan ser reciclados con mayor efectividad, generando un mayor desaprovechamiento de los recursos naturales y un impacto negativo al medio ambiente. 

Por todo lo anteriormente expuesto, surge la necesidad de elaborar estrategias para la aplicación de los principios de las 3R de la ecología en la Urbanización Guaraguao de Puerto La Cruz, con el propósito de incentivar el desarrollo de una cultura del reciclaje y reducir la huella ecológica de los habitantes, planteándose así la siguiente interrogante:

¿Cuáles estrategias se deben implementar para el manejo de los desechos sólidos en la Urbanización Guaraguao de Puerto La Cruz?

\newpage

{\setlength{\parskip}{0cm}
\section{Objetivo General}

Aplicar las 3R de la ecología en el ámbito doméstico para el adecuado manejo de los desechos sólidos en la Urbanización Guaraguao de Puerto La Cruz.
}

{\setlength{\parskip}{0cm}
\section{Objetivos Específicos}

Describir los beneficios que tiene el uso de las 3R de la ecología para el manejo adecuado de los desechos sólidos.
}

Establecer los métodos empleados por la población en la Urbanización Guaraguao de Puerto La Cruz para el manejo de la basura.

Realizar una campaña divulgativa sobre el manejo adecuado de los desechos sólidos para la sensibilización del uso de las 3R de la ecología en la Urbanización Guaraguao de Puerto La Cruz.

Hacer seguimiento de las estrategias propuestas para el empleo de las 3R de la ecología en la Urbanización Guaraguao de Puerto La Cruz.

\newpage

{\setlength{\parskip}{0cm}
\section{Justificación}

La acumulación de los desechos sólidos, debido al manejo inadecuado por parte de las personas, es uno de los principales problemas ambientales y de salud de la actualidad dado el constante crecimiento de la población, y de los patrones de consumo. En la Urbanización Guaraguao de Puerto La Cruz, la basura no solo crea una imagen desagradable en los espacios públicos, sino que también contamina el suelo y el aire, por lo que existe una problemática a la hora de un buen manejo de los desechos por parte de los ciudadanos. 
}

Por ello, es muy importante que los habitantes dispongan de la información pertinente para aplicar medidas que fomenten el consumo responsable y así disminuir los efectos adversos previamente mencionados. Para lograrlo, la presente investigación tiene el objetivo de aplicar los principios de las 3R de la ecología en el ámbito doméstico en la Urbanización Guaraguao de Puerto La Cruz para fomentar la cultura del manejo eficiente de los desechos sólidos.

Los principales beneficios de aplicar las 3R de la ecología en la Urbanización Guaraguao de Puerto La Cruz son: la disminución de los residuos sólidos generados por los habitantes, la mejora de las condiciones del suelo y del aire, y la sensibilización de las personas acerca de la importancia de la implementación de estos principios de la ecología, ya que constituyen una forma de iniciar el cambio antes que el medio ambiente sea contaminado del todo.
 
Además, por ser un tema de actualidad, se espera que esta investigación sirva para ampliar el conocimiento sobre las 3R de la ecología y el manejo adecuado de los desechos sólidos, y aporte algunos criterios a valorar para su aplicación, tanto en futuras investigaciones, así como de manera concreta, en la Urbanización Guaraguao de Puerto La Cruz.

\newpage
